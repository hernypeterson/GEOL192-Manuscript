\chapter{Results}

\section{Mapping Results}

Differentiated lava lows and channels do not uniformly cover the summit of \acf{OM}. Immediately beyond the caldera rim, distinguishable flows are limited mostly to the south-east (e.g., Figure~\ref{fig:uphill-flows}) and north-west margins. Notably, there are essentially zero flows immediately south of the caldera, closer to the modern summit. [FIGURE of all mapped lobate lava flows]. Lava channels are more widespread than lobate flows, especially further from the caldera rim 

\subsection{Mapping Coverage}
\begin{figure}
    \includegraphics[width=\textwidth]{results-coverage.pdf}%
    \caption[Mapping Coverage]{Polygon Flows and Linear Channels}%
    \label{fig:results-coverage}
\end{figure}
% map with polygons for entire study area
% smaller map with polygons, centerlines, and sampled points - points labels with arrows downhill and scaled by color
\subsection{Sampled Points}
\begin{figure}
    \includegraphics[width=\textwidth]{results-samples.pdf}%
    \caption[Sampling Coverage]{Minimum \acf{deform}.}%
    \label{fig:results-samples}
\end{figure}
% map with lines and points for entire study area, same downhill color symbology
% TODO maybe smaller map as well
\subsection{Minimum Tilt}
% map showing each point with a plan view of the calculated axis of minimum tilt, colored by angle of tilt
% do these appear to converge on a single inflation center?
\subsection{Axisymmetric Tilt}
% maps matching previous case in symbology but showing radial tilt instead (so all arrows pointing toward or away from the respective center point)

% scatter plots with radial distance against axisymmetric tilt for a. entire study area and b. sub-area

\section{Model Results}
% figures with radial tilt plotted against radial distance
    % scatter plot for each center point: channel points and flow points
    % line plot with parameter combinations that most closely match scatter plot