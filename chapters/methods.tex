\chapter{Methods}\label{cha:methods}

I first sample paleo- and modern surface attitude data across the summit of \acl{OM} using satellite-derived imagery and topography, respectively. Next, I develop a geometric framework for calculating minimum 3D surface deformation at each sampled location, based on collected attitude data. Next, I determine how much of this deformation must have been toward or away from the center of the edifice for each sampled location. Using these results, I develop a series of axisymmetric numerical models to relate surface deformation to underlying reservoir activity.

\section{Challenges \& Assumptions}
The topographic and structural deformation implies a combination of underlying volcanic processes including a.\ caldera collapse (i.e., magmatic discharge) and b.\ subsequent reservoir inflation (i.e., magmatic recharge). Untangling the superposition of these competing factors is challenging in itself, although preliminary inspection of the study region (Figure~\ref{fig:summit}) indicates that these effects may be localized to a.\ the immediate circumferential region, and b.\ the southern flank, respectively. Additionally, \textit{independent}\footnote{That is, in addition to observed deformation.} stratigraphic constraints on lava flow emplacement relative to the timeline of deformation are unavailable---while crater-dating can be done on lava flows and caldera floors, the timing of reservoir inflation does not leave a temporal record observable from spacecraft. Therefore, certain simplifying assumptions are necessary to make progress with this inquiry. In particular, with only ``initial'' (at flow emplacement) and ``final'' (modern) data points available for each location, only net change can be inferred.

\section{Preparation of Published Data}

\subsection{Raster Data}
The \acf{CTX}\footnote{aboard the \ac{MRO} spacecraft launched in 2005 by \acs{NASA}.} captures \qty{\sim30}{\km} swaths across the entire martian surface in visible $(\lambda=\qtyrange{500}{800}{\nm})$ greyscale at \qty{\sim6}{\m} spatial resolution. \textcite{Dickson2018AGB} blended these swaths to produce a raster mosaic product (hereafter, ``\ac{CTX} mosaic'') which I use to visually identify and map lava flows and flow channels.

The \acf{MOLA}\footnote{aboard the now-retired \ac{MGS} spacecraft launched in 1996 by \acs{NASA}.} returned topography data with horizontal resolution of \qtyproduct{300 x 1000}{\m} at the equator (better at high latitudes) and elevation uncertainty of \qty{\sim3}{\m}~\parencite{smith_mars_2001}. To improve spatial resolution, additional elevation data from the \ac{HRSC}\footnote{aboard the \ac{MEX} spacecraft launched in 2003 by the \ac{ESA}} was blended to product a \ac{DEM} with \qty{200}{\m} pixel resolution. Each pixel's vertical uncertainty is \qty{\sim1}{\m}, with an additional global uncertainty of \qty{\sim1.8}{\m} in the martian areoid (martian equivalent of Earth's geoid). In this project, the global areoid uncertainty is not a concern because only one region (the summit of \ac{OM}) is considered.

These two data sources were registered in an equal-area sinusoidal Mars projection in ArcGIS Pro. The study area is defined by a square \qtyproduct{200 x 200}{\km} centered at the centroid of the outermost \qty{19}{\km} contour,\footnote{This is the highest integer \unit{km} which is roughly circular and completely encloses the caldera complex, implying that it largely records the conical shape of the shield edifice without influence from subsequent caldera collapse or reservoir inflation.} as seen in Figure~\ref{fig:summit}.

\subsection{Prior Mapping}
I also include the caldera region mapping by \textcite{mouginis-mark_geologic_2021}, who identified lobate lava flows Relevant to the current project, 

\section{Study Area Definition \& Preliminary Analysis}

Figure~\ref{fig:summit} shows important topographic patterns at the summit of \ac{OM}. More than \qty{50}{\km} from the center of the figure, topographic contours (\qtyrange{12}{19}{\km}) are fairly regular concentric rings. Closer to the caldera, this radial symmetry breaks down: the caldera complex itself consists of six intersecting collapse pits. On the southern flank, we see a prominent arcuate \qty{20}{\km} contour with the topographic summit (within the \qty{21}{\km} contour) over \qty{20}{\km} from the southern caldera rim. I propose [BASED ON WHAT] that the distal symmetric regions preserve the long-term stable topography of the edifice, while the asymmetric central caldera complex and summit result from relatively recent magmatic activity. 

\subsection{Proto-Edifice Reconstruction}
Therefore, I present a reconstruction of the proto-edifice which interpolates the topography of the distal (beyond outer \qty{19}{\km} contour) regions within the central region. This reconstruction is shown in Figure~(NUMBER). This provides an independent estimation of proto-topography to compare with the estimates based on lava flow misalignment.

\section{Mapping Discordant Features}\label{sec:mapping}

\subsection{Lobate Flows}
I use the \ac{CTX} mosaic to visually identify lava flows near the summit of \ac{OM}. Following \textcite{mouginis-mark_geologic_2021}, I map lobate flow outlines as polygons where possible. From these polygons, I derive centerline features using the \hlss{Polygon To Centerline} tool, as shown in Figure~\ref{fig:flow}.

\subsection{Channels}

Where flow margins are not visible, I map channels directly as linear features. I include discontinuous regions where I infer partial collapse of lava tubes yielding skylight chains,\footnote{The assumption of underlying continuity follows, e.g., \textcite{bleacher_olympus_2007,carr_geologic_2010,peters_lava_2021}.} as shown in Figure~\ref{fig:channel}.

\section{Sampling Discordant Features}

\subsection{Linear Features}

\subsubsection{Segmentation}
[INCOMPLETE] Bounding box, aspect ratio

\subsubsection{Azimuths}

I use the \hlss{Calculate Geometry Attributes} tool to find the \hlss{Line Bearing} attribute for each linear feature (channel or flow centerline), that is, the azimuthal orientation from the start to the end of each feature. This angle is the variable \acf{az1} to each feature, although the following section discussion explains why this variable is not quite ready for use in attitude analysis. While I maintain a consistent ``sense'' in my channel mapping (pointing away from rather than toward the caldera center), the \hlss{Polygon To Centerline} tool does not. Therefore, some flow centerlines need to be reversed using the \hlss{Flip Line} tool.

To determine which centerlines must be reversed, I use the \hlss{Near} tool with \hlss{Angle} selected to determine the azimuth angle from each linear feature to the center of the study area \acs{center}. Here, I invoke the general empirical rule that all flows in the study area point generally \emph{away} from the caldera center. Therefore, a correctly oriented feature is one where \acs{az1} is \ang{\sim180} away from the calculated \hlss{Near Angle}; at the very least, it should be \ang{90} away. Therefore I select features where this angular difference\footnote{Specifically, I reverse features where the cosine of this angular difference is positive. See Appendix~\ref{app:spherical-cosines} for a justification of this approach which addresses an arithmetic issue with angular differences.} is less than \ang{90} to reverse using \hlss{Flip Line}. With all features now correctly oriented away from the caldera center, I recalculate \hlss{Line Bearing}, which establishes the correct \acf{az1} value for attitude analysis. [NEEDS FIGURE HERE]

\subsection{Point Features}

\newcommand{\samplinginterval}{\qty{3}{\km}}

Along each linear feature, I use the \hlss{Generate Points Along Line} tool to create a series of point features where further attitude data collection and analysis will take place. I use \hlss{Distance} mode with sampling interval \samplinginterval\ (rather than \hlss{Percentage} mode) so sampling reflects \emph{coverage}, not feature count. For the same reason, I do not \hlss{Include End Points}; features with length $<\samplinginterval$ are not sampled at all.

\begin{figure}
    \centering
    \begin{subfigure}{\textwidth}
        \centering
        \includegraphics[width=\textwidth]{flow.pdf}
        \caption[Mapped lava flow \& centerline]{A flow with lobate boundaries at each margin is mapped as a polygon (white) from which a linear centerline is derived for sampling.}%
        \label{fig:flow}
    \end{subfigure}
    \begin{subfigure}{\textwidth}
        \centering
        \includegraphics[width=\textwidth]{channel.pdf}
        \caption[Mapped lava channel]{A lava channel is mapped as a linear feature, including regions of discontinuity which are inferred to be collapsed skylight chains over lava tubes.}%
        \label{fig:channel}
    \end{subfigure}
    \caption{Mapping linear features}%
    \label{fig:mapping-linear}
\end{figure}

\begin{figure}
    \centering
    \includegraphics[width=\textwidth]{sampling.pdf}
    \caption[Sampling site selection]{Each linear feature is assigned an average paleo-dip direction. Points are selected for sampling and calculations at \qty{5}{\km} and \qty{3}{\km} intervals for flows and channels, respectively. Paleo-dip direction is assigned to each point from its corresponding line; modern dip and dip direction is assigned to each point from its unique \ac{DEM} neighborhood.}%
    \label{fig:sampling}
\end{figure}

\section{Attitude Data Collection}

\newcommand{\neighborhood}{\qty{2}{\km}}

The first variable collected is \acf{az1}, which each point inherits from its parent linear feature.

Then, I use the \hlss{Surface Parameters} tool on the \ac{MOLA} \ac{DEM} to compute average topographic \hlss{Slope} and \hlss{Aspect} (downhill azimuth) rasters across the entire study area. To avoid capturing local topographic anomalies, these values are averaged\footnote{Technically, a quadratic surface is interpolated over the neighborhood region and dip and dip direction are computed at the center of this surface.} over a ``neighborhood'' with radius \neighborhood. I use the \hlss{Extract Multi Values to Points} tool to assign \acf{sl2} and \acf{az2} to each based on the value of the corresponding raster value at that location. Figure~\ref{fig:surface} shows the compact geometric representation of these attitude parameters as a pole to the topographic surface plane. Figure~\ref{fig:deform-collected} shows a geometric view of all three collected variables, including the incomplete (\acs{az1} but not \acs{sl1}) representation of the paleo-surface attitude.

\begin{figure}
    \floatbox[{\capbeside\thisfloatsetup{floatwidth=sidefil,capbesideposition={left,bottom},capbesidewidth=.55\linewidth}}]{figure}
    {\caption[Pole to plane $\acs{normal2}=(\acs{az2},\acs{sl2})$]{The attitude of a tilted surface (green) is defined relative to horizontal (grey) by its \acf{az2} and \acf{sl2}. \acs{az2} and \acs{sl2} are also the spherical coordinates of a unique unit vector \acs{normal2} which is normal (perpendicular) to the surface. In the subsequent sections, $\acs{normal2}=(\acs{az2},\acs{sl2})$ represents the observed attitude at a sampled location, while $\acs{normal1}=(\acs{az1},\acs{sl1})$ represents the \emph{inferred} paleo-attitude of the same point when the lava flow was emplaced. Azimuth angles increase clockwise from \acf{north} as shown, following geographic convention.}\label{fig:surface}}
    {\begin{tikzpicture}[scale=4.4,tdplot_main_coords]

% origin
\coordinate (O) at (0,0,0);

% also defines (Pxy), (Pxz), (Pyz), etc.
\tdplotsetcoord{P}{\radius}{\ze}{\az}

% fill flat surface
\fill[color = gray!10!white] (0,0) circle (0.5*\radius);

% define tilted surface
\tdplotsetrotatedcoords{\az}{\ze}{0}

% fill tilted surface
\fill[tdplot_rotated_coords, color = green!40!black, opacity=0.4] (0,0) circle (0.4*\radius);

% downhill line
\draw[arrow, tdplot_rotated_coords] (0,0) -- (0:0.4*\radius);

% horizontal surface (front right)
\fill[color = gray!10!white, opacity=0.6] (\az:0.5*\radius) arc (\az:\az+90:0.5*\radius) -- (0,0);

% perpendicular corners
\draw[tdplot_rotated_coords] (0.25,0,0) -- (0.25,0,0.25) -- (0,0,0.25) -- (0,-0.25,0.25) -- (0,-0.25,0);

% horizontal surface (front left)
\fill[color = gray!10!white, opacity=0.6] (\az-90:0.5*\radius) arc (\az-90:\az:0.5*\radius) -- (0,0);

% z axis
\draw[axis] (O) -- (0,0,0.5*\axislength) node[anchor=south]{$z$};

% line az surface
\draw[very thin, dashed,green!40!black] (P) -- (Pxy) -- (O);
\draw[arrow, green!40!black] (O) -- (P) node[anchor = south west] {\acs{normal2}};

% north axis
\draw[axis] (O) -- (0,0.4*\axislength,0) node[anchor=west]{\acs{north}};

% az angle label
\tdplotdrawarc{(O)}{0.4*\radius}{\az}{90}{coordinate, pin={[pin edge={black},-]-60:\acs{az2}}}{}

% az surface
\tdplotsetthetaplanecoords{\az}

% ze angle label
\tdplotdrawarc[tdplot_rotated_coords]{(O)}{0.4*\radius}{0}{\ze}{coordinate, pin={[pin edge={black},-]80:\acs{sl2}}}{}

% ze angle label
\tdplotdrawarc[tdplot_rotated_coords]{(O)}{0.4*\radius}{90}{90+\ze}{coordinate, pin={[pin edge={black},-]180:\acs{sl2}}}{}

\fill[black] (O) circle (0.2pt);

\end{tikzpicture}}
\end{figure}

\section{Attitude Data Calculation}\label{sec:calculation}

As seen in Figure~\ref{fig:deform-collected}, attitude data pertaining to the inferred paleo-surface is incomplete. Namely, while the modern topographic surface attitude is fully characterized by \acs{az2} and \acs{sl2}, only the \acf{az1} can be inferred directly from mapped surface features. Without clear insight into the \acf{sl1} at each sampled point, I present a method for constraining its value under the assumption of \emph{minimum deformation.} 

\subsection{Minimum 3D Tilt from Discordant Flows}\label{sec:3d-deform}
Figure~\ref{fig:deform-calculated} illustrates conceptually my strategy for calculating \acs{sl1} such that the \acf{deform} between \acs{normal2} and \acs{normal1} is minimized. The associated computational method is derived below.
\begin{figure}
    \centering
    \begin{subfigure}{\textwidth}
    \centering
    \begin{tikzpicture}[scale=2.3,tdplot_main_coords]

        % origin
        \coordinate (O) at (0,0,0);
        
        % also defines (Pxy), (Pxz), (Pyz), etc.
        \tdplotsetcoord{P}{\radius}{\ze}{\az}
        \tdplotsetcoord{P'}{\radius}{\zen}{\azi}
        \tdplotsetcoord{Q}{\radius}{90}{\azi}
        
        % grey circle
        \fill[tdplot_main_coords, color = gray!10!white] (0:\radius) arc (0:360:\radius);
        
        % az' surface
        \tdplotsetthetaplanecoords{\azi}
        
        % fill az' surface
        \fill[tdplot_rotated_coords, color = black, opacity = 0.05] (\radius,0) arc (0:90:\radius) -- (0,0);
        
        % line az' surface
        \draw[very thin, dashed] (O) -- (Q);
        \tdplotdrawarc[tdplot_rotated_coords, very thick]{(O)}{\radius}{90}{0}{anchor=north west}{$\theta=\acs{az1}$}
        
        % az surface
        \tdplotsetthetaplanecoords{\az}
        
        % line az surface
        \draw[very thin, dashed] (P) -- (Pxy) -- (O);
        \draw[arrow] (O) -- (P) node[anchor = south west] {\acs{normal2}};
        
        % z-axis
        \draw[axis] (O) -- (0,0,\axislength) node[anchor=south]{$z$};
        
        % |az' - az| angle label
        \tdplotdrawarc{(O)}{0.6}{\azi}{\az+360}{anchor=north}{$\acs{discord}$}
        
        % ze angle label
        \tdplotdrawarc[tdplot_rotated_coords]{(O)}{\radius}{0}{\ze}{anchor=west}{\acs{sl2}}
        

        \fill[black] (O) circle (0.2pt);
    \end{tikzpicture}
    \caption[Attitude data collected]{Attitude data \emph{collected} at each point: \acs{az2}, \acs{sl2}, and \acs{az1}. $\acs{normal2}=(\acs{az2},\acs{sl2})$ as shown in Figure~\ref{fig:surface}. A whole \emph{family} of normal vectors satisfies $\theta=\acs{az1}$. \acs{discord} is the discordance (difference) between the modern \acf{az2} and \acf{az1}.}
    \label{fig:deform-collected}
\end{subfigure}
\begin{subfigure}{\textwidth}
    \centering
    \begin{tikzpicture}[scale=4.3,tdplot_main_coords]

        % origin
        \coordinate (O) at (0,0,0);
        
        % also defines (Pxy), (Pxz), (Pyz), etc.
        \tdplotsetcoord{P}{\radius}{\ze}{\az}
        \tdplotsetcoord{P'}{\radius}{\zen}{\azi}
        \tdplotsetcoord{Q}{\radius}{90}{\azi}
        
        % grey circle
        \fill[tdplot_main_coords, color = gray!10!white] (0:\radius) arc (0:360:\radius);
        
        % az' surface
        \tdplotsetthetaplanecoords{\azi}
        
        % fill az' surface
        \fill[tdplot_rotated_coords, color = red, opacity = 0.3] (\radius,0) arc (0:\zen:\radius) -- (0,0);
        \fill[tdplot_rotated_coords, color = black, opacity = 0.05] (\radius,0) arc (0:90:\radius) -- (0,0);
        
        % line az' surface
        \draw[arrow] (O) -- (P') node[anchor = south east] {\acs{normal1}};
        \draw[very thin, dashed] (O) -- (Q);
        \tdplotdrawarc[tdplot_rotated_coords, very thick]{(O)}{\radius}{90}{0}{anchor=north west}{$\theta=\acs{az1}$}
        \tdplotdrawarc[tdplot_rotated_coords,very thick,red]{(O)}{\radius}{0}{\zen}{anchor=south}{\acs{sl1}}
        
        % az surface
        \tdplotsetthetaplanecoords{\az}
        
        % line az surface
        \draw[very thin, dashed] (P) -- (Pxy) -- (O);
        \draw[arrow] (O) -- (P) node[anchor = south west] {\acs{normal2}};
        
        % z-axis
        \draw[axis] (O) -- (0,0,\axislength) node[anchor=south]{$z$};
        
        % |az' - az| angle label
        \tdplotdrawarc{(O)}{0.6}{\azi}{\az+360}{anchor=north}{$\acs{discord}$}
        
        % ze angle label
        \tdplotdrawarc[tdplot_rotated_coords]{(O)}{\radius}{0}{\ze}{anchor=west}{\acs{sl2}}
        
        % central angle plane
        \tdplotsetrotatedcoords{\azi}{-90+\zen}{0}
        
        % central angle line
        \tdplotdrawarc[tdplot_rotated_coords,blue]{(0,0,0)}{\radius}{\cenang}{0}{coordinate, pin={[pin edge={black},-]-90:\acs{deform}}}{}
        
        % central angle fill
        \fill[tdplot_rotated_coords, color = blue, opacity = 0.1] (\radius,0) arc (0:\cenang:\radius) -- (0,0);
        

        \fill[black] (O) circle (0.2pt);
    \end{tikzpicture}
\caption[Attitude data calculated]{Attitude data \emph{calculated} at each point: \acs{sl1} and \acs{deform}. From the family of vectors with $\theta=\acs{az1}$, the vector \acs{normal1} with zenith angle \acs{sl1} is chosen to minimize the blue \acf{deform} between \acs{normal2} and \acs{normal1}.}
\label{fig:deform-calculated}
\end{subfigure}
    \caption{Calculation of \acf{sl1} and \acf{deform}.}\label{fig:deform}
\end{figure}
Beginning with the spherical law of cosines:\footnote{Derived in Appendix~\ref{app:spherical-cosines}}
\begin{equation}
    \acs{deform}
    =\arccos(\sin\acs{sl1}\sin\acs{sl2}\cos\acs{discord}
    +\cos\acs{sl1}\cos\acs{sl2}).
    \label{eq:central-angle}
\end{equation}
For constant (already measured) \acs{discord} and \acs{sl2}, the value of \acs{sl1} which minimizes \acs{deform} can be calculated by setting:
\begin{equation}
    \frac{\partial}{\partial \acs{sl1}}
    \arccos\left(\sin\acs{sl1}\sin\acs{sl2}\cos\acs{discord}
    +\cos\acs{sl1}\cos\acs{sl2}\right)
    =0.\label{eq:mimimum}
\end{equation}
Differentiating using the chain rule,
\begin{equation}
    \frac{-\,(\cos\ac{sl1}\sin\acs{sl2}
    \cos\acs{discord}-\sin\ac{sl1}\cos\acs{sl2})}
    {\sqrt{1-{(\sin\acs{sl1}\sin\acs{sl2}\cos\acs{discord}
    +\cos\acs{sl1}\cos\acs{sl2})}^2}}
    =0.\label{eq:derivative}
\end{equation}
Multiplying through by the denominator\footnote{The denominator in Equation~\eqref{eq:derivative} is zero when \acs{deform} is \ang{0} or \ang{180}, that is, when \acs{normal2} and \acs{normal1} are equal or antipodal. The $\acs{normal2}=\acs{normal1}$ case behaves as desired in Equation~\eqref{eq:robust-ze'} because $\acs{normal2}=\acs{normal1}\implies\cos\acs{discord}=1\implies\acs{sl1}=\arctan(\tan\acs{sl2}\cdot1)=\acs{sl2}$. The antipodal case is unphysical.} and rearranging to solve for \acs{sl1}:
\begin{gather}
    \cos\ac{sl1}\sin\acs{sl2}
    \cos\acs{discord}-\sin\ac{sl1}\cos\acs{sl2}
    =0,\nonumber\\
    \frac{\sin\acs{sl2}
    \cos\acs{discord}}
    {\cos\acs{sl2}}
    =\frac{\sin\ac{sl1}}{\cos\ac{sl1}},\nonumber\\
    \ac{sl1}
    =\arctan(\tan\ac{sl2}
    \cos\acs{discord}).
    \label{eq:ze'}
\end{gather}
The calculated value of \acs{sl1} can then be substituted back into Equation~\eqref{eq:central-angle} to find \acs{deform}. However, one adjustment must be made to this formula to account for edge cases. Notice in Equation~\eqref{eq:ze'} that when $\cos\acs{discord}$ is negative, so is \acs{sl1}. Since \acs{sl1} ultimately describes a slope, a negative value does not make sense. In fact, this situation corresponds to an \emph{uphill} slope being a closer match with the observed topography than a downhill one, which occurs when the angular distance between the two azimuth values exceeds \ang{90}. This treatment of this case is shown in Figure~\ref{fig:reversal}: negative \acs{sl1} values are reset to zero. Thus, the final equation for \acs{sl1} is:
\begin{equation}
    \boxed{\ac{sl1}
    =\max[0,\arctan(\tan\ac{sl2}
    \cos\acs{discord})].}
    \label{eq:robust-ze'}
\end{equation}

\subsection{Axisymmetric Tilt from Discordant Flows}

Figure~\ref{fig:radial} shows what it means to calculate the component of 3D deformation along the radial-axial plane. Comparison with Figure~\ref{fig:deform-calculated} reveals the geometric similarity between the two problems. Therefore, the same Equation~\ref{eq:ze'} derived in Section~\ref{sec:3d-deform} applies up to small modifications owing to a few important differences. First, there are two distinct blue central angles to be minimized, and thus two distinct red zenith angles to be calculated. The \acf{radial-deform} value to be calculated is the signed difference between the two zenith values, which correspond to the ``projections'' of \acs{normal2} and \acs{normal1} onto the plane containing the $z-$ and $r-$axes. Also, the correction used in Equation~\eqref{eq:robust-ze'} is not desired here, since the vectors \acs{normal2} and \acs{normal1} could just as easily project on either side of the $z-$axis.
\begin{equation}
    \boxed{\acs{radial-deform}
    =\arctan(\tan\ac{sl2}
    \cos[\acs{bearing}-\acs{az2}])
    -\arctan(\tan\ac{sl1}
    \cos[\acs{bearing}-\acs{az1}]).}\label{eq:tilt-from-mapping}
\end{equation}
Note that the order of terms (and thus the sign of the difference) is significant. For example, \ang{-5} of radial deformation would represent a ``caving-in'' toward the caldera, while \ang{+5} of radial deformation would be more like an outward bulge.


\begin{figure}
    \floatbox[{\capbeside\thisfloatsetup{floatwidth=sidefil,capbesideposition={left,top},capbesidewidth=.55\linewidth}}]{figure}
    {\caption[Reversal case: $\acs{discord}>\ang{90}$]{When $\acs{discord}>\ang{90}$, Equation~\eqref{eq:ze'} gives $\acs{sl1}<0$, which does not make physical sense as a downhill slope in the \acs{az1} direction. Instead, \acs{sl1} is set to zero, the \emph{non-negative} downhill paleo-slope that minimizes \acs{deform}.}\label{fig:reversal}}
    {\hspace*{-0.4\linewidth}\begin{tikzpicture}[scale=4.2,tdplot_main_coords]

        % origin
        \coordinate (O) at (0,0,0);
        
        % also defines (Pxy), (Pxz), (Pyz), etc.
        \tdplotsetcoord{P}{\radius}{\ze}{\azrev}
        \tdplotsetcoord{P'}{\radius}{\zen}{\azi}
        \tdplotsetcoord{Q}{\radius}{90}{\azi}
        \tdplotsetcoord{-Q}{\radius}{90}{180+\azi}
        
        % grey circle
        \fill[tdplot_main_coords, color = gray!10!white] (O) circle (0.6);
        
        % az' surface
        \tdplotsetthetaplanecoords{\azi}
        
        % fill az' surface
        \fill[tdplot_rotated_coords, color = red, opacity = 0.3] (\radius,0) arc (0:-\zen:\radius) -- (0,0);
        \fill[tdplot_rotated_coords, color = black, opacity = 0.05] (\radius,0) arc (0:90:\radius) -- (0,0);
        \fill[tdplot_rotated_coords, color = black, opacity = 0.05] (\radius,0) arc (0:-90:\radius) -- (0,0);
        
        % line az' surface
        % \draw[arrow] (O) -- (P') node[anchor = south east] {\acs{normal1}};
        \draw[very thin, dashed] (-Q) -- (Q);
        \tdplotdrawarc[tdplot_rotated_coords, very thick]{(O)}{\radius}{90}{0}{anchor=north west}{$\theta=\acs{az1}$}
        \tdplotdrawarc[tdplot_rotated_coords, dashed]{(O)}{\radius}{0}{-\zen}{}{}
        % \tdplotdrawarc[tdplot_rotated_coords,dashed]{(O)}{\radius}{0}{-90}{coordinate, pin={[pin edge={black},-]-90:$\acs{sl2}=\acs{deform}$}}{}
        \tdplotdrawarc[tdplot_rotated_coords,dashed]{(O)}{\radius}{0}{-90}{coordinate, pin={[pin edge={black,solid},-]90:$\theta=\acs{az1}+\ang{180}$}}{}
        \tdplotdrawarc[tdplot_rotated_coords,dashed,red]{(O)}{\radius}{0}{-\zen}{}{}
        
        
        % az surface
        \tdplotsetthetaplanecoords{\azrev}
        
        % line az surface
        \draw[very thin, dashed] (P) -- (Pxy) -- (O);
        \draw[arrow] (O) -- (P) node[anchor = west] {\acs{normal2}};
        
        % z-axis
        \draw[axis] (O) -- (0,0,\axislength) node[anchor=south]{$z=\acs{normal1}$};
        
        % |az' - az| angle label
        \tdplotdrawarc{(O)}{0.6}{\azi}{\azrev-360}{anchor=north}{\acs{discord}}
        
        % ze angle label
        \tdplotdrawarc[tdplot_rotated_coords]{(O)}{\radius}{0}{\ze}{coordinate, pin={[pin edge={black},-]-90:$\acs{sl2}=\acs{deform}$}}{}
        
        % central angle plane
        \tdplotsetrotatedcoords{\azi}{-90-\zen}{0}
        
        % central angle line
        \tdplotdrawarc[tdplot_rotated_coords,blue, dashed]{(0,0,0)}{\radius}{\cenang}{0}{}{}
        
        % central angle fill
        \fill[tdplot_rotated_coords, color = blue, opacity = 0.1] (\radius,0) arc (0:\cenang:\radius) -- (0,0);
        

        \fill[black] (O) circle (0.3pt);
    \end{tikzpicture}}
    \vspace*{-9em}
    \floatbox[{\capbeside\thisfloatsetup{floatwidth=sidefil,capbesideposition={right,bottom},capbesidewidth=.6\linewidth}}]{figure}
    {\caption[Calculation of \acf{radial-deform} from mapping]{Geometric view of \acl{radial-deform}. \acs{bearing} is the bearing angle from the center of the study area \acs{center} to the sampled location, measured clockwise from \acf{north}. It defines the radial-axial plane onto which \acs{normal1} and \acs{normal2} are projected using \acs{deform}-minimizing procedure developed in Section~\ref{sec:3d-deform}. The resulting projected angles \acs{proj1} and \acs{proj2} are not labeled to reduce visual clutter. Their difference $\acs{proj2}-\acs{proj1}=\acs{radial-deform}$ is the component of tilt in the direction radial to the proto-edifice center \acs{center}.}\label{fig:radial}}
    {
    \begin{tikzpicture}[scale=7,tdplot_main_coords]

        % origin
        \coordinate (O) at (0,0,0);
        
        % also defines (Pxy), (Pxz), (Pyz), etc.
        \tdplotsetcoord{P}{\radius}{\ze}{\az}
        \tdplotsetcoord{C}{-0.9*\axislength}{90}{\az-45}
        \tdplotsetcoord{P'}{\radius}{\zen}{\azi}
        \tdplotsetcoord{Q}{\radius}{90}{\azi}
        \tdplotsetcoord{Z}{\radius}{90}{\THETA}
        \tdplotsetcoord{Zend}{\radius}{30}{\THETA}
        
        % horizontal circumference
        % \tdplotdrawarc[tdplot_main_coords]{(O)}{0.45*\radius}{0}{360}{}{}

        % grey circle
        \fill[tdplot_main_coords, color = gray!10!white] (O) circle (0.3);

        % horizontal circle caldera center
        \fill[color = gray!10!white] (C) circle (0.3);

        % north arrow from caldera center
        \draw[axis] (C) --+ (0,0.3*\axislength,0) node[anchor=west]{\acs{north}};

        % bearing angle labels
        \tdplotdrawarc{(C)}{0.3}{\THETA}{90}{anchor=north west}{\acs{bearing}}

        % az' surface
        \tdplotsetthetaplanecoords{\azi}
        
        \tdplotdrawarc[tdplot_rotated_coords]{(O)}{\radius}{0}{\zen}{anchor=south}{\acs{sl1}}

        \draw[very thin, dashed] (P') -- (P'xy) -- (O);
        \draw[arrow] (O) -- (P') node[anchor = south east] {\acs{normal1}};
        
        % THETA surface
        \tdplotsetthetaplanecoords{\THETA}

        % fill THETA surface
        \fill[tdplot_rotated_coords, color = black, opacity = 0.05] (-30:\radius) arc (-30:50:\radius) -- (0,0.76604444*\radius) -- (0,-0.5*\radius);
        \fill[tdplot_rotated_coords, color = red, opacity = 0.3] (\zenproj:\radius) arc (\zenproj:\zeproj:\radius) -- (0,0);     

        % line THETA surface
        \tdplotdrawarc[arrow, tdplot_rotated_coords]{(O)}{\radius}{-30}{50}{}{}
        
        % projected def-radial
        \tdplotdrawarc[arrow, tdplot_rotated_coords,red]{(O)}{\radius}{\zenproj}{\zeproj}{coordinate, pin={[pin edge={red,-},pin distance = 1cm,-]0:\acs{radial-deform}}}{}
        
        % az surface
        \tdplotsetthetaplanecoords{\az}
        
        % line az surface
        \draw[very thin, dashed] (P) -- (Pxy) -- (O);
        \draw[arrow] (O) -- (P) node[anchor = south west] {\acs{normal2}};
        
        % z-axis
        \draw[axis] (O) -- (0,0,\axislength) node[anchor=south]{$z$};
        \draw[axis] (C) -- (\THETA:0.76604444*\axislength) node[anchor=north east]{$r$};
        
        % az angle labels
        \tdplotdrawarc{(O)}{0.3}{\azi}{\THETA+360}{anchor=north east}{$\acs{az1}-\acs{bearing}$}
        \tdplotdrawarc{(O)}{0.3}{\THETA}{\az}{anchor=north}{$\acs{bearing}-\acs{az2}$}
        
        % ze angle label
        \tdplotdrawarc[tdplot_rotated_coords]{(O)}{\radius}{0}{\ze}{anchor=west}{\acs{sl2}}
        
        % central angle plane for n
        \tdplotsetrotatedcoords{\THETA}{-90+\zeproj}{0}
        
        % central angle line n
        \tdplotdrawarc[tdplot_rotated_coords, blue]{(0,0,0)}{\radius}{\zecenang}{0}{}{}

        % central angle fill n'
        \fill[tdplot_rotated_coords, color = blue, opacity = 0.05] (0:\radius) arc (0:\zecenang:\radius) -- (0,0);

        % central angle plane for n'
        \tdplotsetrotatedcoords{\THETA}{-90+\zenproj}{0}
        
        % central angle line n'
        \tdplotdrawarc[tdplot_rotated_coords, blue]{(0,0,0)}{\radius}{-\zencenang}{0}{}{}
        
        % central angle fill n'
        \fill[tdplot_rotated_coords, color = blue, opacity = 0.05] (-\zencenang:\radius) arc (-\zencenang:0:\radius) -- (0,0);

        \fill[black] (O) circle (0.2pt);
        \fill[black] (C) circle (0.2pt) node[anchor=south]{\acs{center}};
        
        \end{tikzpicture}}
\end{figure}

\section{Modelling Reservoir Pressure Change}

I use the numerical modelling software COMSOL Multiphysics 6.1 (COMSOL) to construct a numerical representation of \acf{OM}. The model is axisymmetric, about the center point discussed previously as the center of the \qty{19}{\km} contour. A profile from due south of this point is used as the 

\subsection{Axisymmetric Tilt from Modelled Displacement}

The numerical formula for \acf{radial-deform} given \acs{disp_a} and \acs{disp_b} is derived from Figure~\ref{fig:tilt-from-model}:
\begin{equation}
    \boxed{\acs{radial-deform} = 
    \arctan\left(\dfrac{-\acs{dz1}-\acs{ddisp_z}}{\acs{dr1}+\acs{ddisp_r}}\right) - \arctan\left(\frac{-\acs{dz1}}{\acs{dr1}}\right).}\label{eq:tilt-from-model}
\end{equation}
For a flat model ($\acs{dz1} = 0$), Equation~\eqref{eq:tilt-from-model} reduces to
\begin{equation}
    \boxed{\acs{radial-deform} = 
    \arctan\left(\dfrac{-\acs{ddisp_z}}{\acs{dr1}+\acs{ddisp_r}}\right).}\label{eq:tilt-from-flat-model}
\end{equation}

\begin{figure}
    \newcommand{\uar}{3.5}
\newcommand{\uaz}{11}
\newcommand{\ubr}{5.5}
\newcommand{\ubz}{4.5}
\newcommand{\dz}{2.5}
\newcommand{\dr}{8}

\begin{tikzpicture}[scale=0.95]
    \coordinate (orig) at (0,0);
    \coordinate (A1) at (0,\dz);
    \coordinate (B1) at (\dr,0);
    
    \path (A1) + (\uar, \uaz) coordinate (A2);
    \path (B1) + (\ubr, \ubz) coordinate (B2);

    \path (A1) + (\uar - \ubr, \uaz - \ubz) coordinate (A1_trans);

    \draw[-latex] (orig) --+ (-0.5,0) node[anchor=east] {\acs{center}};

    \draw[] (orig) -- node[fill=white] {\acs{dr1}} (B1);
    \draw[] (orig) -- node[fill=white] {$-\acs{dz1}$} (A1);

    \draw[ultra thick] (A1) -- node[sloped,fill=white] {Original Surface} (B1);

    % % main displacement vectors
    \draw[arrow,gray] (A1) -- node[sloped, fill=white] {\acs{disp_a}} (A2);
    \draw[arrow,gray] (B1) -- node[sloped, fill=white] {\acs{disp_b}} (B2);
    
    % % \fill[blue, opacity = 0.15] (B1) --++ (162.7:2.5) arc (162.7:180:2.5);
    % % \path (B1) + (172:2) node {\acs{proj1}};
    
    \draw[arrow,gray,dashed] (A1) -- node[fill=white] {$-\acs{ddisp}$} (A1_trans);
    
    \draw[arrow,gray,dashed] (A2) --node[fill=white] {$\acs{disp_b}$} (A1_trans);
    
    % % \fill[red, opacity = 0.3] (B2) --++ (138:2.75) arc (138:162.7:2.75);
    % % \path (B2) + (150:2) node {\acs{tilt}};

    % % Draw displaced surface lines
    \draw[ultra thick] (A1_trans) -- node[sloped,fill=white] {Displaced Surface (Translated)} (B1);
    \draw[ultra thick] (A2) -- node[sloped,fill=white] {Displaced Surface} (B2);
    
    % % draw points
    \fill (B1) circle (2pt) node[anchor = north] {$B_1$};
    \fill (A2) circle (2pt) node[anchor = south] {$A_2$};
    \fill (B2) circle (2pt) node[anchor = west] {$B_2$};
    \fill (A1_trans) circle (2pt);
    \fill (A1) circle (2pt) node[anchor = east] {$A_1$};
    
\end{tikzpicture}
    \caption[Calculation of \acf{radial-deform} from modelling]{Axisymmetric tilt from finite element displacement. A model element at the surface of the model is shown as the ``Original Surface'' with slope \acs{proj1}. During the model, the two vertices of the surface element $A_1$ and $B_1$ are displaced via $\acs{disp_a}$ and $\acs{disp_b}$ to $A_2$ and $B_2$ respectively, bringing the element edge to the ``Deformed surface'' with slope \acs{proj2}. The horizontal (radial) and vertical components of the difference $\acs{disp_b} - \acs{disp_a} = \acs{ddisp}$ are labeled. The difference $\acs{proj2}-\acs{proj1}=\acs{radial-deform}$ is the exact same quantity illustrated in Figure~\ref{fig:radial}. While Equation~\eqref{eq:tilt-from-mapping} arrives at this value using lava flows discordant to their observed topographic surroundings, Equation~\eqref{eq:tilt-from-model} determines the value via numerically modelled surface displacement due to underlying reservoir inflation presumed to be responsible.}%
    \label{fig:tilt-from-model}
\end{figure}

\subsection{Comparison with Analytical Model}
The discrete Equation~\eqref{eq:tilt-from-flat-model} can be taken to the continuous limit by dividing each term in the numerator and denominator by the element width \acs{dr1}:
\begin{equation}
\acs{radial-deform}
    = \lim_{\acs{dr1}\to0} 
    \arctan\left(\dfrac{-\acs{ddisp_z}/\acs{dr1}}{\acs{dr1}/\acs{dr1}
    + \acs{ddisp_r}/\acs{dr1}}\right) = 
    \arctan\left(\dfrac{-\acs{disp_z'}}{1+\acs{disp_r'}}\right),\label{eq:tilt-from-flat-analytical-model}
\end{equation}
where $'$ denotes the derivative with respect to $r_1$. I use the \textcite{mogi_relations_1958} Model:
\begin{gather}
    \acs{disp_z} = ad{(d^2+r^2)}^{-1.5},\label{eq:uz_mogi}\\
    \acs{disp_r} = ar{(d^2+r^2)}^{-1.5},\label{eq:ur_mogi}\\
    a = {3R^3\Delta P}/{4G},\nonumber
\end{gather}
where $R$ is the reservoir radius, $\Delta P$ is the overpressure, and $G$ is the elastic shear modulus. Differentiating Equations~\eqref{eq:uz_mogi} and~\eqref{eq:ur_mogi}, substituting into Equation~\eqref{eq:tilt-from-flat-analytical-model}, and simplifying:
\begin{equation}
    \boxed{\acs{radial-deform} = \arctan\left(\frac{3adr}{{(d^2+r^2)}^{2.5}+ad^2-2ar^2}\right).}\label{eq:tilt-from-mogi-model}
\end{equation}

\subsection{Model Parameter Space}

