\chapter{Introduction}
\ac{OM} is the largest volcano in the solar system (or anywhere else known to science), with a diameter of roughly \qty{600}{\km} (Figure~\ref{fig:edifice}) and a summit elevation of over \qty{20}{\km}~\parencite{plescia_morphometric_2004}. The striking observation which catalyzed the current inquiry is that numerous lava flows around its summit caldera appear to flow uphill~\parencite[Figure~\ref{fig:uphill-flows}; after][]{mouginis-mark_late-stage_2019}. Similar discordance between lava flow direction and topography are observed in the flexural basin surrounding the main edifice, with horizontal discordance between flow direction and regional dip direction of up to \ang{\sim15}~\parencite{chadwick_late_2015}, rather than the \ang{\sim180} discordance observed at the summit. In both cases, however, these discrepancies suggest a history of surface deformation that occurred \textit{after} the lava flows were emplaced in what was once a downhill direction. The aim of the current project is to quantify this surface deformation and assess the underlying processes responsible.

\begin{figure}
    \centering
    \includegraphics[width=\textwidth]{edifice.pdf}
    \caption[\acl{OM}]{The \ac{OM} edifice is \qty{\sim600}{\km} in diameter and its summit on the southern flank is more than \qty{20}{\km} high. It features a nested caldera complex more than \qty{50}{\km} in diameter and \qty{\sim2}{\km} deep. Main map is a hillshade (shaded relief) derived from \acs{MOLA} elevation data; inset of entire martian surface is \acs{MOLA} elevation. Contours in \unit{\km}.}\label{fig:edifice}
\end{figure}

\begin{figure}
    \centering
    \includegraphics[width=\textwidth]{uphill-flows.pdf}
    \caption[Discordant lava flows]{Several lava flows (white) in the southern flank region appear to flow uphill toward the modern summit. \acs{CTX} mosaic basemap. Contours in \unit{\km}. After Figure 3 in \textcite{mouginis-mark_late-stage_2019}.}%
    \label{fig:uphill-flows}
\end{figure}

\section{Significance of Caldera-Forming Systems}

\subsection{Assessing Hazards}
\subsection{Understanding Fundamental Reservoir Behavior}

\section{Planetary Perspective}

\subsection{Broader View of Volcanism}
\subsection{Inquiries Unique to Mars}
The study of \ac{OM} contributes to a more comprehensive understanding of volcanic processes than could be achieved with a solely Earth-based inquiry. Compared to Earth, the unique challenges of studying martian volcanoes are obvious: the inability to make in-situ measurements and the lack of evidence for ongoing eruptive activity stand out. 

However, there are important advantages as well. Most notably, surface deformation signatures are well preserved in the martian environment over hundreds of millions of years, especially at high elevations where weathering and erosion are far less active. This allows for a more comprehensive picture of deformation history than is commonly available on Earth, where large caldera-forming edifices are rare and often poorly preserved. Additionally, the unique martian conditions that yield the largest volcanoes in the solar system broaden our perspective of how volcanism works in general.

\section{Physical Volcanism of \acl{OM}}
\subsection{First Observations}
\subsection{Modelling}
\subsection{Mapping}
\subsection{Dating}
\subsection{Relationship with Tharsis Montes}

The volcanic history of Mars differs from that of Earth in many ways. On Earth, dominant volcanic systems include mid-ocean ridges, island \& continental arcs, hotspot chains, and \acp{LIP}. Ridges and arcs have no clear analogues on Mars; hotspots and \acp{LIP} share many similarities with martian features but reflect differing thermal, mechanical, and chemical properties.

The martian geologic timescale is different from the terrestrial one; one of the most obvious differences is that the martian scale has finer gradations \textit{early} in its history, (within the first \qty{\sim1.5}{Gyr}) reflecting initial bombardment and dynamic volcanic, geophysical, \& sedimentary processes, while later eons have been more stable and quiescent.

The most precise absolute dating methods using radioisotopes are largely unavailable because few martian samples exist on Earth. Nonetheless, radiochronometry of martian meteorites has revealed volcanic eruptions within the last \qty{\sim1.5}{Gyr}~\parencite[e.g.,][]{cohen_taking_2017}. For assigning dates to specific surfaces, researchers count accumulated craters, which can often constrain absolute ages to within a factor of \num{\sim2}~\parencite{kneissl_map-projection-independent_2011}.

From absolute dating and stratigraphic relationships, three major periods are identified~\parencite{carr_geologic_2010}. First, the Noachian period lasted roughly from \qtyrange{4.1}{3.7}{Ga}. During this period, volcanic activity was common everywhere on Mars. The Tharsis Plateau and \ac{OM} edifices may have been largely constructed by the end of the Noachian~\parencite{isherwood_volcanic_2013}. Next, the Hesperian period \qtyrange{3.7}{3.1}{Ga} saw the end of \ac{LHB}, intermittent flooding but overall reduced prevalence of liquid water, and initial concentration of volcanism within Tharsis and Elysium volcanic provinces.\footnote{\ac{OM} is distinct from but closely associated with the Tharsis province~\parencite{bleacher_trends_2007}.} Finally, the ongoing Amazonian period (\qty{\sim3.1}{Ga}--present) has been characterized by intermittent localized volcanism within the major provinces, limited fluvial activity, and slow sedimentary processes.

\begin{figure}
    \centering
    \includegraphics[width=\textwidth]{summit.pdf}
    \caption[Study area: \acl{OM} summit]{Study area at the summit of \acl{OM} (inset). Sinusoidal Martian Projection. Contours in \unit{km}. Square is \qtyproduct{200 x 200}{\km}, centered at the midpoint of the outermost \qty{19}{\km} contour.}\label{fig:summit}
\end{figure}

\section{Research Questions}

The following questions will inform the research project: 
\begin{enumerate}
    \item can discordant lava flows be used as reliable planetary ``tiltmeters'' to provide insight into subsurface activity?
    \item what patterns of regional surface deformation do discordant lava flows imply?
    \item what combinations of reservoir processes and geometries best explain inferred surface deformation?
\end{enumerate}

\section{Hypotheses}

I suspect:
\begin{enumerate}
    \item Recognizing the incompleteness of underlying attitude information, discordant lava flows will be able to provide limited insight into subsurface activity.
    \item Discordant flows will reveal limited surface deformation restricted to the immediate vicinity of the arcuate crest south of the caldera complex.
    \item Inflation under the southern crest will account for observed tilt better than subsidence at the caldera complex itself. 
\end{enumerate}

% \begin{figure}
    % \centering
    % \includegraphics[width=\textwidth]{mars.pdf}
    % \caption[Martian topography]{\acs{MOLA}-derived global topography of Mars. \acl{OM} lies exactly on the central meridian, between northern lowlands (green) and southern highlands (orange). Elevation in \unit{\m}. Horizontal scale distances computed at equator.}
    % \label{fig:mars}
% \end{figure}